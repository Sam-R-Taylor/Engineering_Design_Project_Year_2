\documentclass[10pt,twoside]{article}

\newcommand{\reporttitle}{Mars Rover}
\newcommand{\reportauthor}{}
\newcommand{\reporttype}{Project Report}


% include files that load packages and define macros
\input{includes} % various packages needed for maths etc.
\input{notation} % short-hand notation and macros

\begin{document}

% Last modification: 2016-09-29 (Marc Deisenroth)
\begin{titlepage}

\newcommand{\HRule}{\rule{\linewidth}{0.5mm}} % Defines a new command for the horizontal lines, change thickness here


%----------------------------------------------------------------------------------------
%	LOGO SECTION
%----------------------------------------------------------------------------------------


    \begin{tikzpicture}[remember picture,overlay]
    \node[anchor=north west,inner sep=1cm] at (current page.north west)
          {\includegraphics[width = 4cm]{./figures/imperial}\\[0.5cm] };
    \end{tikzpicture}


\begin{center} % Center remainder of the page

%----------------------------------------------------------------------------------------
%	HEADING SECTIONS
%----------------------------------------------------------------------------------------
\textsc{\LARGE \reporttype}\\[1.5cm] 
\textsc{\Large Imperial College London}\\[0.5cm] 
\textsc{\large Department of Electrical and Electronic Engineering}\\[0.5cm] 
%----------------------------------------------------------------------------------------
%	TITLE SECTION
%----------------------------------------------------------------------------------------

\HRule \\[0.4cm]
{ \huge \bfseries \reporttitle}\\ % Title of your document
\HRule \\[1.5cm]
\end{center}
%----------------------------------------------------------------------------------------
%	AUTHOR SECTION
%----------------------------------------------------------------------------------------

%\begin{minipage}{0.4\hsize}
\begin{center}
\textbf{Authors:}\\
Sam Taylor - 01705109\\
Martin Prusa - 01713176\\
Leonardo Garofalo - 01746454\\
Katherine Zhang - 01504365\\
Maximus Wickham - 01717673\\
Matilde Piccoli - 01764158 \\
\end{center}
\vspace{2cm}
\makeatletter
\centering
Date: \@date 

\vfill % Fill the rest of the page with whitespace



\makeatother


\end{titlepage}



\newpage

\tableofcontents

\newpage
\section{Abstract}
\section{Project Management}
\subsection{Project Structure}
The given timeline for the project was 5 weeks. The first week was allocated for research and planning. Each member carried out research on their own module and a meeting was held to identify the main tasks and create a tentative project timeline. From the second week on-wards, bi-weekly meetings were held: one meeting at the beginning of the week to set the targets for that week and one meeting in the middle of the week to check up on progress. Week 2 was planned to work on the basic functionalities for each module so that a very basic prototype of the rover could be put together. In week 3, more sophisticated implementations and additional functions were added. By the end of week 4, the design of each module was finalised and integration of the whole rover was completed. The final week of the project was used to clean up any issues with the integrated rover and make final adjustments. 

\subsection{Meeting Structure}
At the beginning of each week we held a 45 minute meeting. Each member would take 5 minutes to update the group on their progress on the targets set for the previous week, discuss any issues they found and set new targets for the week. Midway through the week we had a brief 20 minute meeting to check in on everyone’s progress and give the opportunity for the members to bring up any problems or difficulties they might have ran into and adjust plans accordingly.

\subsection{Tools Used for Project Management}
A Gantt chart was used to illustrate the main tasks to be completed for each module and set deadlines for their completion. It served as an effective method for tracking progress and keeping on track. 
The productivity software Notion was also used for each member to make notes of their initial design process, development and testing. It provided an easy method to share progress with the rest of the team. The shared calendar was used to set dates of meetings and mark internal deadlines. 
Git was used for version control.

\vfill

\section{Design Process / Planning}
\subsection{Problem Definition and Design Criteria}

\subsubsection{Drive}

\subsubsection{Energy}

\subsubsection{Vision}

\subsubsection{Command}

\subsubsection{Control}

\subsubsection{Integration}

\subsection{Design Overview / Implementation Strategy}

\subsubsection{Drive}

\subsubsection{Energy}

\subsubsection{Vision}

\subsubsection{Command}

\subsubsection{Control}

\subsubsection{Integration}

\section{Development and Implementation}

\subsection{Drive}

\subsection{Energy}

\subsection{Vision}

\subsection{Command}

\subsection{Control}

\subsubsection{Integration}

\section{Testing and Evaluation}
\subsection{Testing and analysis}

\subsubsection{Drive}

\subsubsection{Energy}

\subsubsection{Vision}

\subsubsection{Command}

\subsubsection{Control}

\subsubsection{Integration}

\subsection{Critical Analysis / Evaluation}

\subsubsection{Drive}

\subsubsection{Energy}

\subsubsection{Vision}

\subsubsection{Command}

\subsubsection{Control}

\subsubsection{Integration}

\section{Reflection}

\appendix











\end{document}
